\documentclass{article}
\usepackage{amsmath}
\usepackage{algorithm}
\usepackage{algpseudocode}
\usepackage{listings, xcolor}
\usepackage{paralist}
\definecolor{dkgreen}{rgb}{0,0.6,0}
\definecolor{gray}{rgb}{0.5,0.5,0.5}
\definecolor{mauve}{rgb}{0.58,0,0.82}

\lstset{frame=tb,
  language=Java,
  aboveskip=3mm,
  belowskip=3mm,
  showstringspaces=false,
  columns=flexible,
  basicstyle={\small\ttfamily},
  numbers=none,
  numberstyle=\tiny\color{gray},
  keywordstyle=\color{blue},
  commentstyle=\color{dkgreen},
  stringstyle=\color{mauve},
  breaklines=true,
  breakatwhitespace=true
  tabsize=3
}

\title{Project 2 - Support Multiprogramming}
\author{Xie Yuanhang\\ 2011012344\and
Kuang Zhonghong\\ 2011012357\and
Li Qingyang\\ 2011012360\and
Yin Mingtian\\ 2011012362\and
Wang Qinshi\\ 2012011311}
\date{}
\begin{document}
\maketitle
\tableofcontents{}
\section{URL of git Repository}
\texttt{https://github.com/bcip/nachos}
\section{Implement File System Calls}
\subsection{Overview}
In this task, we mainly need to implement system calls related to file system such as \texttt{create, open, read, write, close} and \texttt{unlink}
.
\subsection{Correctness Constraints}
\begin{enumerate}
	\item[$\bullet$] \texttt{handleCreate: }
		\begin{compactitem}
		\item Call the method \texttt{open(String name, boolean true} provided in \texttt{FileSystem}.
		\item Save the returned \texttt{Openfile} by above method.
		\end{compactitem}
	\item[$\bullet$] \texttt{handleOpen:} Same as \texttt{handleCreate} but call method \texttt{open(String name, boolean true} provided in \texttt{FileSystem}.
	\item[$\bullet$] \texttt{handleRead, handleWrite:} Keep a record for the location of your reading.
	\item[$\bullet$] \texttt{handleClose:} Note that a given file
descriptor can be reused if the file associated with it is closed, and that different processes can
use the same file descriptor (i.e. integer) to refer to different files.
	\item[$\bullet$] \texttt{handleUnlink:} Need to judge whether there is still a process have the file open.
	\item[$\bullet$] \texttt{handleHalt:} Need to judge whether the caller is the root process.
\end{enumerate}
\subsection{Declaration}
\begin{compactitem}
\item Declare an enclapsed \text{FileDescriptor} class recording the \texttt{OpenFile}, location, a \texttt{boolean} \texttt{remove}.
\item Declare an array with length 16 of \texttt{FileDescriptor} \texttt{fileDescriptor}, prepare two \texttt{Openfile}: standart input and output.
\item \texttt{Create:} 
	\begin{enumerate}
		\item[$\bullet$] read file name from virtual memory.
		\item[$\bullet$] call \texttt{open(String name, true)} and save the returned \texttt{Openfile}.
		\item[$\bullet$] save the \texttt{fileDescriptor} into the array \texttt{fileDescriptor}.
	\end{enumerate}
\item \texttt{Open:} Almost the same as \texttt{Create} but call \texttt{open(String name, false}.
\item \texttt{Read:}
	\begin{compactitem}
		\item Read the index of \texttt{FileDescriptor} ,the writing location of virtual memory \texttt{location} and the length of characters \texttt{count}.
		\item Pick the corresponding \texttt{FileDescriptor} and get the \texttt{Openfile} file, and call the method \texttt{file.read} to write the message into an array \texttt{buffer}.
		\item Write \texttt{buffer} into virtual memory, update the location.
	\end{compactitem}
\item \texttt{Write:}
	\begin{compactitem}
		\item Read the index of \texttt{FileDescriptor} ,the writing location of virtual memory \texttt{location} and the length of characters \texttt{count}.
		\item Pick the corresponding \texttt{FileDescriptor} and get the \texttt{Openfile} file, and read the message from virtual memory into an array \texttt{buffer}.
		\item Write \texttt{buffer} into the file, update the location.
	\end{compactitem}
\item \texttt{Close:}
	\begin{compactitem}
		\item Pick the corresponding \texttt{FileDescriptor}.
		\item Close the file.
		\item Decide whether to remove the file i.e. whether it is unlinked.
	\end{compactitem}
\item \texttt{Unlink:}
	\begin{compactitem}
		\item Read fileName and find corresponding file.
		\item If any processes still have the file open, the file will remain in existence until the last file descriptor referring to it is closed. 
		\item If no process still have the file open, simply remove it.
	\end{compactitem}
\end{compactitem}
\subsection{Description}
\subsection{Test}

\section{Implement Support for Multiprogramming}
\subsection{Overview}
\subsection{Correctness Constraints}
\subsection{Declaration}
\subsection{Description}
\subsection{Test}

\section{Implement Other System Calls}
\subsection{Overview}
\subsection{Correctness Constraints}
\subsection{Declaration}
\subsection{Description}
\subsection{Test}

\section{Implement Lottery Scheduler}
\subsection{Overview}
\subsection{Correctness Constraints}
\subsection{Declaration}
\subsection{Description}
\subsection{Test}
\end{document} 
