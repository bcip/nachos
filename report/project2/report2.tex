\documentclass{article}
\usepackage{amsmath}
\usepackage{algorithm}
\usepackage{algpseudocode}
\usepackage{listings, xcolor}
\usepackage{paralist}
\usepackage{chngpage}

\definecolor{dkgreen}{rgb}{0,0.6,0}
\definecolor{gray}{rgb}{0.5,0.5,0.5}
\definecolor{mauve}{rgb}{0.58,0,0.82}

\lstset{frame=tb,
  language=Java,
  aboveskip=3mm,
  belowskip=3mm,
  showstringspaces=false,
  columns=flexible,
  basicstyle={\small\ttfamily},
  numbers=none,
  numberstyle=\tiny\color{gray},
  keywordstyle=\color{blue},
  commentstyle=\color{dkgreen},
  stringstyle=\color{mauve},
  breaklines=true,
  breakatwhitespace=true
  tabsize=3
}

\title{Project 2 - Support Multiprogramming}
\author{
Xie Yuanhang  \\   2011012344\and
Kuang Zhonghong  \\   2011012357\and
Li Qingyang   \\   2011012360 \and
Yin Mingtian   \\   2011012362\and
Wang Qinshi   \\   2012011311}

\changetext{}{+1cm}{-0.5cm}{-0.5cm}{}

\date{}
\begin{document}
\maketitle
\tableofcontents{}
\section{URL of git Repository}
\texttt{https://github.com/bcip/nachos}
 \section{Implement File System Calls}
\subsection{Overview}
In this task, we mainly need to implement system calls related to file system such as \texttt{create, open, read, write, close} and \texttt{unlink}
.
\subsection{Correctness Constraints}
\begin{compactitem}
\item Implement \texttt{creat, open, read, write, close} and \texttt{unlink}.
\item Bullet-proof those system calls.
\item \texttt{halt} can only be called from root process.
\item Each open file must have unique fiel descriptor and including standart input and output.
\item Be able to open 16 files simultaneously in a single process.
\end{compactitem}
\subsection{Declaration}
\begin{compactitem}
\item Declare new class \texttt{FileManager} in \texttt{UserKernel}. Remember each file's name, how many process open it, and whether it should 
	be unlinked.
\item Declare an enclapsed \text{FileDescriptor} class recording the \texttt{OpenFile}, location.
\item Declare an array with length 16 of \texttt{FileDescriptor} \texttt{fileDescriptor}, prepare two \texttt{Openfile}: standart input and output.
\item \texttt{Create:}
	\begin{enumerate}
		\item[$\bullet$] read file name from virtual memory.
		\item[$\bullet$] call \texttt{open(String name, true)} and save the returned \texttt{Openfile}.
		\item[$\bullet$] save the \texttt{fileDescriptor} into the array \texttt{fileDescriptor}.
	\end{enumerate}
\item \texttt{Open:} Almost the same as \texttt{Create} but call \texttt{open(String name, false}.
\item \texttt{Read:}
	\begin{compactitem}
		\item Read the index of \texttt{FileDescriptor} ,the writing location of virtual memory \texttt{location} and the length of characters \texttt{count}.
		\item Pick the corresponding \texttt{FileDescriptor} and get the \texttt{Openfile} file, and call the method \texttt{file.read} to write the message into an array \texttt{buffer}.
		\item Write \texttt{buffer} into virtual memory, update the location.
	\end{compactitem}
\item \texttt{Write:}
	\begin{compactitem}
		\item Read the index of \texttt{FileDescriptor} ,the writing location of virtual memory \texttt{location} and the length of characters \texttt{count}.
		\item Pick the corresponding \texttt{FileDescriptor} and get the \texttt{Openfile} file, and read the message from virtual memory into an array \texttt{buffer}.
		\item Write \texttt{buffer} into the file, update the location.
	\end{compactitem}
\item \texttt{Close:}
	\begin{compactitem}
		\item Pick the corresponding \texttt{FileDescriptor}.
		\item Close the file.
		\item Decide whether to remove the file i.e. whether it is unlinked.
	\end{compactitem}
\item \texttt{Unlink:}
	\begin{compactitem}
		\item Read fileName and find corresponding file.
		\item If any processes still have the file open, the file will remain in existence until the last file descriptor referring to it is closed.
		\item If no process still have the file open, simply remove it.
	\end{compactitem}
\end{compactitem}
\subsection{Description}
In the form of pseudocode,
\begin{lstlisting}
class UserKernel {
	class FileManager {
		class FileNode {
			String fileName;
			int times;
			boolean unlink;
		}
		public FileManager() {
			new a List of Map fileMapList between fileName and FileNode
		}
	}
}
\end{lstlisting}
\begin{algorithm}
		\caption{In class UserProcess}
    \begin{algorithmic}
        \Procedure{handleHalt()}{}
            \If{this process is not "root" process}
                \State return -1
            \EndIf
						\State make \texttt{Machine.halt} call.
        \EndProcedure
    \end{algorithmic}
    \begin{algorithmic}
        \Procedure{handleCreate}{int file}
            \State filename $\leftarrow$ get the name of file from virtual memory
            \State temp $\leftarrow$ UserKernel.fileSystem.open(filename, true)
            \If{temp is null}
                \State return -1
            \Else
                \State find the empty element in the array of \texttt{FileDescriptor}
                \If{no empty element}
                    \State return -1
                \Else
									\State write the \texttt{FileDescriptor} in \texttt{fileDescriptor}.
									\State add this file into \texttt{UserKernel.FileManager}
                   \State return the index
                \EndIf
            \EndIf
        \EndProcedure
    \end{algorithmic}
\end{algorithm}
\begin{algorithm}
    \begin{algorithmic}
        \Procedure{handleOpen}{int file}
            \State filename $\leftarrow$ get the name of file from virtual memory
            \State temp $\leftarrow$ UserKernel.fileSystem.open(filename, false)
            \If{temp is null}
                \State return -1
            \Else
                \State find the empty element in \texttt{fileDescriptor}
                \If{no empty element}
                    \State return -1
                \Else
								\State write the \texttt{FileDescriptor} in \texttt{fileDescriptor}
										\State add the file into \texttt{fileMapList} \texttt{UserKernel.fileManager} if not exists,
										\State if exists, only increase \texttt{file.times} by 1.
                    \State return the index
                \EndIf
            \EndIf
        \EndProcedure
    \end{algorithmic}
    \begin{algorithmic}
        \Procedure{handleRead}{int index, int vaddr, int size}
            \If{index is out of range and element of this index is null}
                \State return -1
            \EndIf
            \State read the file to buffer with capacity size
            \If{the read above is not success}
                \State return -1
            \Else
                \State write buffer to vitural memory
                \State update position of the file
            \EndIf
        \EndProcedure
    \end{algorithmic}
    \begin{algorithmic}
        \Procedure{handleWrite}{int index, int vaddr, int size}
            \If{index is out of range and element of this index is null}
                \State return -1
            \EndIf
            \State write to buffer from virtual memory
            \State write to file
            \If{the write above is not success}
                \State return -1
            \Else
                \State update position of the file
            \EndIf
        \EndProcedure
    \end{algorithmic}
\end{algorithm}
\begin{algorithm}
    \begin{algorithmic}
        \Procedure{handleClose}{int index}
            \If{the index is out of range}
                \State return -1
            \EndIf
						\State close the file at the index 
						\State reduce \texttt{fileNode.times} by 1 in \texttt{UserKernel.fileManager}
						\State if \texttt{fileNode.times} turns to be 0, remove from the \texttt{fileMapList}.
						\State if \texttt{fileNode.times} equals 0 and \texttt{fileNode.unlink} is \texttt{true}, unlink the file.
            \If{this file is denote as remove}
                \If{rest of \texttt{FileDescriptor} doesn't contain this file}
                    \State remove this file
                \EndIf
            \EndIf
        \EndProcedure
    \end{algorithmic}
    \begin{algorithmic}
        \Procedure{handleUnlink}{int file}
            \State filename $\leftarrow$ get the name of file from virtual memory
						\If{this filename is not contain in \texttt{UserKernel.fileManager}}
                \State remove this file
            \Else
						\State find \texttt{fileNode} of \texttt{file} in \texttt{UserKernel.fileManager}.
						\State mark \texttt{fileNode.unlink} as \texttt{true} in \texttt{UserKernel.fileManager} 
            \EndIf
        \EndProcedure
    \end{algorithmic}
\end{algorithm}

\subsection{Test}
Test with c code through MIPS. We can call methods \texttt{close, creat, unlink, write, read, open} in a c file with different orders and see if 
the operations meet the requirements.
\begin{compactitem}
\item Normal create, close, write, read, open, unlink
\item Test Bullet-proof: close, write, read, unlink before file exists
\item Test Bullet-proof: close, write, read before open
\item Test Bullet-proof: test with invalid parameters.
\end{compactitem}
\section{Implement Support for Multiprogramming}
\subsection{Overview}
In this task, we mainly need to implement support for multiprogramming i.e. alter \texttt{readVirtualMemory, writeVirtualMemory, loadSections,
unloadSections} to make sure that memory for processes are used properly and efficiently.
\subsection{Correctness Constraints}
\begin{compactitem}
\item implement system calss \texttt{writeVirtualMemory, readVirtualMemory, load, unload}.
\item efficiently use memory.
\item COFF's \texttt{TranslationEntry} should be readonly.
\item Above methods should not throw exception when fail, but to return the number of bytes transferred.
\item set up pageTable structure with load sections.
\end{compactitem}
\subsection{Declaration}
\begin{compactitem}
\item Declare a list of available physical pages \texttt{availablePages} in \texttt{UserKernel} and a \texttt{Lock} lock for synchronization.
\item Declare an array of \texttt{TranslationEntry} \texttt{pageTable} in \texttt{UserProcess}.
\item \texttt{readVirtualMemory:} transfer data from virtual memory into array \texttt{data}. Check the range.
\item \texttt{writeVirtualMemory:} transfer data from array \texttt{data} into virtual memory. Check the range.
\item \texttt{loadSections:} set the pageTable structure which is an array of \texttt{TranslationEntry}.
\item \texttt{unloadSections:} release the resources allocated by \texttt{loadSections}.
\end{compactitem}
\subsection{Description}
\begin{algorithm*}
	\caption{In \texttt{class UserKernel}}
	\begin{algorithmic}
		\Procedure {UserKernel.initialize}{String[] args}
			\State \dots
			\State lock = new Lock()
			\While{availablePages.size less than numPhypages}
				\State \texttt{availablePages.add} new page
			\EndWhile
		\EndProcedure
	\end{algorithmic}
\end{algorithm*}
\begin{algorithm*}
	\caption{In \texttt{class UserProcess}}
	\begin{algorithmic}
		\Procedure {readVirtualMemory}{int vaddr, byte[] data, int offset, int length}
			\State validate address and other parameters
			\State \texttt{amount} $\leftarrow$ 0
			\State Get start and end page number by \texttt{Machine.processor.pageFromAddress}
			\For{from state page to end page}
				\If{current page is invalid}
					\State \Return \texttt{amount}
				\EndIf
				\State copy the corresponding piece of data into right page by \texttt{System.arraycopy}
				\State increase \texttt{amount}, change \texttt{offset}
			\EndFor
			\State \Return number of bytes successfully transferred
		\EndProcedure
		\Procedure{writeVirtualMemory}{int vaddr, byte[] data, int offset, int length}
			\State validate address and other parameters
			\State \texttt{memory} $\leftarrow$ \texttt{Machine.processor.getMemory}.
			\State Get start and end page number by \texttt{Machine.processor.pageFromAddress}
			\State \texttt{/** forloop similar to readVirtualMemory but with inverse direction **/ }
			\For{from start page to end page}
				\State copy the corresponding piece of data into right page
			\EndFor
			\State \Return number of bytes successfully transferred
		\EndProcedure
		\Procedure{loadSections}{}
			\State Acquire \texttt{UserKernel.lock}
			\State Get available physical memory from \texttt{UserKernel.availablePages} by remove items from \texttt{availablePages}
			\State and store into \texttt{pageTable}
			\State (i.e. map physical memory into virtual memory).
			\For{each section \texttt{s} and virtual page number \texttt{n}}
				\State \texttt{m}$leftarrow$\texttt{availablePages.first}
				\State \texttt{TranslationEntry[n]}$\leftarrow$\texttt{(n,m,true,false,false,false)}.
			\EndFor
			\State Release \texttt{UserKernel.lock}
		\EndProcedure
		\Procedure{unloadSections}{}
			\State Acquire \texttt{UserKernel.lock}
			\State Release the allocated memory by adding them back into \texttt{availablePages}.
			\State Release \texttt{UserKernel.lock}
			\State close all files.
		\EndProcedure
	\end{algorithmic}
\end{algorithm*}
\subsection{Test}
Set method \texttt{selfTest} in \texttt{UserKernel}.
\begin{compactitem}
\item Test allocation and deallocation of pages.
\item Test whether read virtual memeory and write virtual memory function correctly.
\item Test load and unload a \texttt{coff} file which allocates a large piece of memory by creating a \texttt{coff} file.
\end{compactitem}
\section{Implement Other System Calls}
\subsection{Overview}
In this task, we mainly need to implement the rest system calls such as \texttt{exec, join, exit} to support operations on thread.
\subsection{Correctness Constraints}
\begin{compactitem}
\item \texttt{handleExit}:
\begin{compactitem}[$\bullet$]
\item its thread should be terminated, and the process clean up any state associated with it.
\item last process should cause the machine to halt.
\end{compactitem}
\item \texttt{handleExec}:
\begin{compactitem}[$\bullet$]
\item exec file should exist and be coff file.
\item exec arguments should be legal.
\end{compactitem}
\item \texttt{handleJoin}: only a process's parent can join to it.
\item process ID should be a globally unique positive integer.
\item Bullet-proof those system calls.
\end{compactitem}
\subsection{Declaration}
\begin{compactitem}
\item Declare a static number \texttt{processIdCounter} and a lock \texttt{processIdCounterLock} for it. And in the construct method, we add \texttt{processIdCounter} by 1 with the lock around.
\item Declare a array of \texttt{UThread} \texttt{threadList} of the process. When a thread is created, it should be added into its process's 
	\texttt{threadList}.
\item Declare \texttt{parent}, \texttt{UserProcess} class. The parent process of the process.
\item Declare \texttt{childList}, a list of child process.
\item Declare \texttt{exitStatusMap} and a lock \texttt{exitStatusMapLock}. To save the exitvalue of its child process.
\end{compactitem}

\subsection{Description}
In the form of pseudocode,
%\begin{lstlisting}
%class UserKernel {
%	class ProcessManager {
%		class ProcessNode {
%			Process process;
%			ProcessNode parent;
%			int pid;
%		}
%		public ProcessManager() {
%			new a List of Map processMapList between process ID and process
%		}
%	}
%}
%\end{lstlisting}
\begin{algorithm*}
    \begin{algorithmic}
				\Procedure{UserProcess()}{}
					\State do initialize stuff\dots
					\State acquire \texttt{processIdCounterLock}
					\State increase \texttt{processIdCounter} by 1
					\State release \texttt{processIdCounterLock}
%					\State add itself into the list in \texttt{UserKernel.processManager}
				\EndProcedure
				\Procedure{execute}{String name, String[] args}
					\State validata parameters
					\State new an \texttt{UThread} of this process.
					\State add the thread into \texttt{threadList}.
				\EndProcedure
        \Procedure{handleExit}{int exitvalue}
            \State remove this from its parent's \texttt{childList}
						\State unload this process's section by \texttt{unloadSections}
            \State transfer exitvalue to its parent
            \State clear this process's \texttt{exitStatusMap} and \texttt{childList}
            \If{this process is "root" process}
                \State Kernel.kernel.terminate()
            \Else
								\State release all its memory, close all its files.
%								\State remove itself from the list in \texttt{UserKernel.processManager}
                \State finish this process
            \EndIf
        \EndProcedure
		\end{algorithmic}
\end{algorithm*}
\begin{algorithm*}
	\begin{algorithmic}
				\Procedure{handleExec}{int fileNameAddress,  int argc, int argv}
					\State Some bullet-proof statements\dots
					\For{\texttt{argc} times}
						\State Calculate argument address
						\State Read each argument
					\EndFor
					\State New a \texttt{UserProcess} \texttt{child} and execute it with arguments return a boolean \texttt{executed}
					\If{\texttt{executed} is true}
						\State add \texttt{child} into \texttt{childList} and set \texttt{child.parent} as \texttt{this}
					\Else 
						\State \Return -1
					\EndIf
					\State \Return \texttt{child.ProcessId}.
				\EndProcedure
				\Procedure{handleJoin}{int pid, int statusAddress}
					\State find the corresponding child process \texttt{child} using \texttt{pid}
					\For{each thread in \texttt{child.threadList}}
						\State call \texttt{thread.join}
					\EndFor
					\State once \texttt{child} return, remove \texttt{child} from \texttt{childList}
					\State acquire \texttt{exitStatusMapLock}
					\State get \texttt{child}'s exit status from \texttt{exitStatusMap}
					\State release \texttt{exitStatusMapLock}
					\State write exit status into \texttt{statusAddress}.
				\EndProcedure
    \end{algorithmic}
\end{algorithm*}
\subsection{Test}

\section{Implement Lottery Scheduler}
\subsection{Overview}
In this task, we mainly need to implement \texttt{LotteryScheduler} by extending \texttt{PrioriryScheduler}.
\subsection{Correctness Constraints}
\begin{itemize}
    \item
        Functions \texttt{getPriority}, \texttt{getEffectivePriority}, \texttt{setPriority},
        \texttt{ThreadQueue.waitForAccess}, \texttt{ThreadQueue.acquire} and
        \texttt{ThreadQueue.nextThread} can be called only when \texttt{Machine.interrupt().disabled()}.
    \item
        Function \texttt{setPriority(KThread thread, int priority)} can be called only when
        \texttt{priority >= priorityMinimum} and \texttt{priority <= priorityMaximum}.
    \item
        The total number of tickets must fit the range of \texttt{int}.
\end{itemize}
\subsection{Declaration}
\begin{itemize}
    \item
        Over ride constants \texttt{priorityDefault}, \texttt{priorityMinimum} and \texttt{priorityMaximum}.
    \item
        Ove rride functions \texttt{newThreadQueue} and \texttt{getThreadState}.
    \item
        Class \texttt{LotteryQueue} extends \texttt{PriorityQueue}:
        \begin{itemize}
            \item 
                Override functions
                \begin{itemize}
                    \item \texttt{pickNextThread}
                    \item \texttt{updateDonatingPriority}
                \end{itemize}
        \end{itemize}
    \item 
        Class \texttt{ThreadState} extends \texttt{super.ThreadState}:
        \begin{itemize}
            \item
                Override function \texttt{updateEffectivePriority}
        \end{itemize}
\end{itemize}
\subsection{Description}
\begin{algorithm*}
	\caption{In class LotteryScheduler}
	\begin{algorithmic}
		\Procedure{newThreadQueue}{boolean transferPriority}
			\State \Return new LotteryQueue(transferPriority)
		\EndProcedure
		\Procedure{getThreadState}{KThread thread}
		\EndProcedure
	\end{algorithmic}
\end{algorithm*}
\begin{algorithm*}
	\caption{In class LotteryQueue}
	\begin{algorithmic}
		\Procedure{pickNextThread}{}
		\EndProcedure
		\Procedure{updateDonatingPriority}{}
		\EndProcedure
	\end{algorithmic}
\end{algorithm*}
\begin{algorithm*}
	\caption{In class LotteryScheduler.ThreadState}
	\begin{algorithmic}
		\Procedure{updateEffectivePriority}{}
		\EndProcedure
	\end{algorithmic}
\end{algorithm*}
\subsection{Test}
\end{document}
