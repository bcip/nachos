\documentclass{article}
\usepackage{amsmath}
\usepackage{algorithm}
\usepackage{algpseudocode}
\usepackage{listings, xcolor}
\usepackage{paralist}
\usepackage{chngpage}

\definecolor{dkgreen}{rgb}{0,0.6,0}
\definecolor{gray}{rgb}{0.5,0.5,0.5}
\definecolor{mauve}{rgb}{0.58,0,0.82}

\lstset{frame=tb,
  language=Java,
  aboveskip=3mm,
  belowskip=3mm,
  showstringspaces=false,
  columns=flexible,
  basicstyle={\small\ttfamily},
  numbers=none,
  numberstyle=\tiny\color{gray},
  keywordstyle=\color{blue},
  commentstyle=\color{dkgreen},
  stringstyle=\color{mauve},
  breaklines=true,
  breakatwhitespace=true
  tabsize=3
}

\title{Project 2 - Support Multiprogramming}
\author{
Xie Yuanhang  \\   2011012344\and
Kuang Zhonghong  \\   2011012357\and
Li Qingyang   \\   2011012360 \and
Yin Mingtian   \\   2011012362\and
Wang Qinshi   \\   2012011311}

\changetext{}{+1cm}{-0.5cm}{-0.5cm}{}

\date{}
\begin{document}
\maketitle
\tableofcontents{}
\section{URL of git Repository}
\texttt{https://github.com/bcip/nachos}
\section{Implement File System Calls}
\subsection{Overview}
In this task, we mainly need to implement system calls related to file system such as \texttt{create, open, read, write, close} and \texttt{unlink}
.
\subsection{Correctness Constraints}
\begin{enumerate}
	\item[$\bullet$] \texttt{handleCreate: }
		\begin{compactitem}
		\item Call the method \texttt{open(String name, boolean true} provided in \texttt{FileSystem}.
		\item Save the returned \texttt{Openfile} by above method.
		\end{compactitem}
	\item[$\bullet$] \texttt{handleOpen:} Same as \texttt{handleCreate} but call method \texttt{open(String name, boolean true} provided in \texttt{FileSystem}.
	\item[$\bullet$] \texttt{handleRead, handleWrite:} Keep a record for the location of your reading.
	\item[$\bullet$] \texttt{handleClose:} Note that a given file
descriptor can be reused if the file associated with it is closed, and that different processes can
use the same file descriptor (i.e. integer) to refer to different files.
	\item[$\bullet$] \texttt{handleUnlink:} Need to judge whether there is still a process have the file open.
	\item[$\bullet$] \texttt{handleHalt:} Need to judge whether the caller is the root process.
\end{enumerate}
\subsection{Declaration}
\begin{compactitem}
\item Declare an enclapsed \text{FileDescriptor} class recording the \texttt{OpenFile}, location, a \texttt{boolean} \texttt{remove}.
\item Declare an array with length 16 of \texttt{FileDescriptor} \texttt{fileDescriptor}, prepare two \texttt{Openfile}: standart input and output.
\item \texttt{Create:}
	\begin{enumerate}
		\item[$\bullet$] read file name from virtual memory.
		\item[$\bullet$] call \texttt{open(String name, true)} and save the returned \texttt{Openfile}.
		\item[$\bullet$] save the \texttt{fileDescriptor} into the array \texttt{fileDescriptor}.
	\end{enumerate}
\item \texttt{Open:} Almost the same as \texttt{Create} but call \texttt{open(String name, false}.
\item \texttt{Read:}
	\begin{compactitem}
		\item Read the index of \texttt{FileDescriptor} ,the writing location of virtual memory \texttt{location} and the length of characters \texttt{count}.
		\item Pick the corresponding \texttt{FileDescriptor} and get the \texttt{Openfile} file, and call the method \texttt{file.read} to write the message into an array \texttt{buffer}.
		\item Write \texttt{buffer} into virtual memory, update the location.
	\end{compactitem}
\item \texttt{Write:}
	\begin{compactitem}
		\item Read the index of \texttt{FileDescriptor} ,the writing location of virtual memory \texttt{location} and the length of characters \texttt{count}.
		\item Pick the corresponding \texttt{FileDescriptor} and get the \texttt{Openfile} file, and read the message from virtual memory into an array \texttt{buffer}.
		\item Write \texttt{buffer} into the file, update the location.
	\end{compactitem}
\item \texttt{Close:}
	\begin{compactitem}
		\item Pick the corresponding \texttt{FileDescriptor}.
		\item Close the file.
		\item Decide whether to remove the file i.e. whether it is unlinked.
	\end{compactitem}
\item \texttt{Unlink:}
	\begin{compactitem}
		\item Read fileName and find corresponding file.
		\item If any processes still have the file open, the file will remain in existence until the last file descriptor referring to it is closed.
		\item If no process still have the file open, simply remove it.
	\end{compactitem}
\end{compactitem}
\subsection{Description}
Shown in pseudocode.
\begin{algorithm}
    \begin{algorithmic}
        \Procedure{handleHalt()}{}
            \If{this process is not "root" process}
                \State return -1
            \EndIf
            \State make the \texttt{halt} system call
        \EndProcedure
    \end{algorithmic}
    \begin{algorithmic}
        \Procedure{handleCreate}{int file}
            \State filename $\leftarrow$ get the name of file from virtual memory
            \State temp $\leftarrow$ UserKernel.fileSystem.open(filename, true)
            \If{temp is null}
                \State return -1
            \Else
                \State find the empty element in the array of \texttt{FileDescriptor}
                \If{no empty element}
                    \State return -1
                \Else
                    \State write the \texttt{FileDescriptor}
                    \State return the index
                \EndIf
            \EndIf
        \EndProcedure
    \end{algorithmic}
\end{algorithm}
\begin{algorithm}
    \begin{algorithmic}
        \Procedure{handleOpen}{int file}
            \State filename $\leftarrow$ get the name of file from virtual memory
            \State temp $\leftarrow$ UserKernel.fileSystem.open(filename, false)
            \If{temp is null}
                \State return -1
            \Else
                \State find the empty element in the array of \texttt{FileDescriptor}
                \If{no empty element}
                    \State return -1
                \Else
                    \State write the \texttt{FileDescriptor}
                    \State return the index
                \EndIf
            \EndIf
        \EndProcedure
    \end{algorithmic}
    \begin{algorithmic}
        \Procedure{handleRead}{int index, int vaddr, int size}
            \If{index is out of range and element of this index is null}
                \State return -1
            \EndIf
            \State read the file to buffer with capacity size
            \If{the read above is not success}
                \State return -1
            \Else
                \State write buffer to vitural memory
                \State update position of the file
            \EndIf
        \EndProcedure
    \end{algorithmic}
    \begin{algorithmic}
        \Procedure{handleWrite}{int index, int vaddr, int size}
            \If{index is out of range and element of this index is null}
                \State return -1
            \EndIf
            \State write to buffer from virtual memory
            \State write to file
            \If{the write above is not success}
                \State return -1
            \Else
                \State update position of the file
            \EndIf
        \EndProcedure
    \end{algorithmic}
\end{algorithm}
\begin{algorithm}
    \begin{algorithmic}
        \Procedure{handleClose}{int index}
            \If{the index is out of range}
                \State return -1
            \EndIf
            \State close the file at the index
            \If{this file is denote as remove}
                \If{rest of \texttt{FileDescriptor} doesn't contain this file}
                    \State remove this file
                \EndIf
            \EndIf
        \EndProcedure
    \end{algorithmic}
    \begin{algorithmic}
        \Procedure{handleUnlink}{int file}
            \State filename $\leftarrow$ get the name of file from virtual memory
            \If{this filename is not contain in the array of \texttt{FileDescriptor}}
                \State remove this file
            \Else
                \State denote the elements in array of \texttt{FileDescriptor} equal filename as remove
            \EndIf
        \EndProcedure
    \end{algorithmic}
\end{algorithm}

\subsection{Test}

\section{Implement Support for Multiprogramming}
\subsection{Overview}
In this task, we mainly need to implement support for multiprogramming i.e. alter \texttt{readVirtualMemory, writeVirtualMemory, loadSections,
unloadSections} to make sure that memory for processes are used properly and efficiently.
\subsection{Correctness Constraints}
\begin{compactitem}
\item Efficiency of memory: we can't simply allocate pages in a contiguous block and we need to make sure that a process's memory is freed on exit.
\item Maintaince of pageTable for user process. Methods
\end{compactitem}
\subsection{Declaration}
\begin{compactitem}
\item Declare a list of available physical pages \texttt{availablePages} in \texttt{UserKernel} and a \texttt{Lock} lock for synchronization.
\item Declare an array of \texttt{TranslationEntry} \texttt{pageTable} in \texttt{UserProcess}.
\item \texttt{readVirtualMemory:} transfer data from virtual memory into array \texttt{data}. Check the range.
\item \texttt{writeVirtualMemory:} transfer data from array \texttt{data} into virtual memory. Check the range.
\item \texttt{loadSections:} set the pageTable structure which is an array of \texttt{TranslationEntry}.
\item \texttt{unloadSections:} release the resources allocated by \texttt{loadSections}.
\end{compactitem}
\subsection{Description}
\begin{algorithm*}
	\caption{In \texttt{class UserKernel}}
	\begin{algorithmic}
		\Procedure {UserKernel.initialize}{String[] args}
			\State \dots
			\State lock = new Lock()
			\State Allocation the required number of physical pages into \texttt{availablePages}
		\EndProcedure
	\end{algorithmic}
\end{algorithm*}
\begin{algorithm*}
	\caption{In \texttt{class UserProcess}}
	\begin{algorithmic}
		\Procedure {readVirtualMemory}{int vaddr, byte[] data, int offset, int length}
			\State Get start and end page number by \texttt{Machine.processor.pageFromAddress}
			\For{from state page to end page}
				\State copy the corresponding piece of data into right page by \texttt{System.arraycopy}
				\State and set the \texttt{TranslateEntry}.
			\EndFor
		\EndProcedure
		\Procedure{writeVirtualMemory}{int vaddr, byte[] data, int offset, int length}
			\State Get the reference to memory by \texttt{Machine.processor.getMemory}.
			\State Get start and end page number by \texttt{Machine.processor.pageFromAddress}
		\EndProcedure
		\Procedure{loadSections}{}
			\State Acquire \texttt{UserKernel.lock}
			\State Get available physical memory from \texttt{UserKernel.allPages} by remove items from \texttt{availablePages}
			\State and store into \texttt{pageTable}
			\State Release \texttt{UserKernel.lock}
			\State (i.e. map physical memory into virtual memory).
			\For{each section}
				\State load the section into memory through \texttt{TranslationEntry}
			\EndFor
		\EndProcedure
		\Procedure{unloadSections}{}
			\State Acquire \texttt{UserKernel.lock}
			\State Release the allocated memory by adding them back into \texttt{availablePages}.
			\State Release \texttt{UserKernel.lock}
			\State close all files.
		\EndProcedure
	\end{algorithmic}
\end{algorithm*}
\subsection{Test}

\section{Implement Other System Calls}
\subsection{Overview}
In this task, we mainly need to implement the rest system calls such as \texttt{exec, join, exit} to support operations on thread.
\subsection{Correctness Constraints}
\begin{compactitem}
\item \texttt{handleExit}:
\begin{compactitem}[$\bullet$]
\item its thread should be terminated, and the process clean up any state associated with it.
\item last process should cause the machine to halt.
\end{compactitem}
\item \texttt{handleExec}:
\begin{compactitem}[$\bullet$]
\item exec file should exist and is coff file.
\item exec arguments should legal.
\end{compactitem}
\item \texttt{handleJoin}: only a process's parent can join to it.
\end{compactitem}
\subsection{Declaration}
\begin{compactitem}
\item Declare a static number \texttt{processIdCounter} and a lock \texttt{processIdCounterLock} for it. And in the construct method, we add \texttt{processIdCounter} by 1 with the lock around.
\item Declare \texttt{thread} of the process, \texttt{UThread} class.
\item Declare \texttt{parent}, \texttt{UserProcess} class. The parent process of the process.
\item Declare \texttt{childList}, a list of child process.
\item Declare \texttt{exitStatusMap} and a lock \texttt{exitStatusMapLock}. To save the exitvalue of its child process.
\end{compactitem}

\subsection{Description}
\begin{algorithm*}
    \begin{algorithmic}
				\Procedure{UserProcess()}{}
					\State do initialize stuff\dots
					\State acquire \texttt{processIdCounterLock}
					\State increase \texttt{processIdCounter} by 1
					\State release \texttt{processIdCounterLock}
				\EndProcedure
				\Procedure{execute}{String name, String[] args}
					\State new an \texttt{UThread} of this process.
					\State remember the thread into \texttt{thread}.
				\EndProcedure
        \Procedure{handleExit}{int exitvalue}
            \State remove this from its parent's \texttt{childList}
						\State unload this process's section by \texttt{unloadSections}
            \State transfer exitvalue to its parent
            \State clear this process's \texttt{exitStatusMap} and \texttt{childList}
            \If{this process is "root" process}
                \State Kernel.kernel.terminate()
            \Else
                \State finish this process
            \EndIf
        \EndProcedure
				\Procedure{handleExec}{int fileNameAddress,  int argc, int argv}
					\State Some bullet-proof statements\dots
					\For{\texttt{argc} times}
						\State Calculate argument address
						\State Read each argument
					\EndFor
					\State New a \texttt{UserProcess} \texttt{child} and execute it with arguments return a boolean \texttt{executed}
					\State If \texttt{executed} is true, add \texttt{child} into \texttt{childList} and set \texttt{child.parent} as \texttt{this}; else return -1
					\State \Return \texttt{child.ProcessId}.
				\EndProcedure
				\Procedure{handleJoin}{int pid, int statusAddress}
					\State find the corresponding child process \texttt{child} using \texttt{pid}
					\State call \texttt{child.thread.join}
					\State once \texttt{child.thread.join} return, remove \texttt{child} from \texttt{childList}
					\State acquire \texttt{exitStatusMapLock}
					\State get \texttt{child}'s exit status from \texttt{exitStatusMap}
					\State release \texttt{exitStatusMapLock}
					\State write exit status into \texttt{statusAddress}.
				\EndProcedure
    \end{algorithmic}
\end{algorithm*}
\subsection{Test}

\section{Implement Lottery Scheduler}
\subsection{Overview}
In this task, we mainly need to implement \texttt{LotteryScheduler} by extending \texttt{PrioriryScheduler}.
\subsection{Correctness Constraints}
\begin{itemize}
    \item
        Functions \texttt{getPriority}, \texttt{getEffectivePriority}, \texttt{setPriority},
        \texttt{ThreadQueue.waitForAccess}, \texttt{ThreadQueue.acquire} and
        \texttt{ThreadQueue.nextThread} can be called only when \texttt{Machine.interrupt().disabled()}.
    \item
        Function \texttt{setPriority(KThread thread, int priority)} can be called only when
        \texttt{priority >= priorityMinimum} and \texttt{priority <= priorityMaximum}.
    \item
        The total number of tickets must fit the range of \texttt{int}.
\end{itemize}
\subsection{Declaration}
\begin{itemize}
    \item
        Over ride constants \texttt{priorityDefault}, \texttt{priorityMinimum} and \texttt{priorityMaximum}.
    \item
        Ove rride functions \texttt{newThreadQueue} and \texttt{getThreadState}.
    \item
        Class \texttt{LotteryQueue} extends \texttt{PriorityQueue}:
        \begin{itemize}
            \item 
                Override functions
                \begin{itemize}
                    \item \texttt{pickNextThread}
                    \item \texttt{updateDonatingPriority}
                \end{itemize}
        \end{itemize}
    \item 
        Class \texttt{ThreadState} extends \texttt{super.ThreadState}:
        \begin{itemize}
            \item
                Override function \texttt{updateEffectivePriority}
        \end{itemize}
\end{itemize}
\subsection{Description}
\begin{algorithm*}
	\caption{In class LotteryScheduler}
	\begin{algorithmic}
		\Procedure{newThreadQueue}{boolean transferPriority}
			\State \Return new LotteryQueue(transferPriority)
		\EndProcedure
		\Procedure{getThreadState}{KThread thread}
		\EndProcedure
	\end{algorithmic}
\end{algorithm*}
\begin{algorithm*}
	\caption{In class LotteryQueue}
	\begin{algorithmic}
		\Procedure{pickNextThread}{}
		\EndProcedure
		\Procedure{updateDonatingPriority}{}
		\EndProcedure
	\end{algorithmic}
\end{algorithm*}
\begin{algorithm*}
	\caption{In class LotteryScheduler.ThreadState}
	\begin{algorithmic}
		\Procedure{updateEffectivePriority}{}
		\EndProcedure
	\end{algorithmic}
\end{algorithm*}
\subsection{Test}
\end{document}
